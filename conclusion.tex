\chapter{Conclusion}
\label{chap:conclusion}

In this thesis, we explored a fundamental question in topological data analysis:
\emph{when are the results we get reliable?} Specifically, we looked at the
stability of generalised density functions. If a small change in our input data
leads to a large change in the resulting persistence diagram, we can't be
confident in our analysis. Our main goal was to find and describe classes of
GDFs that don't have this problem, functions that are stable against
perturbations of the input point cloud.

\section{Summary of findings}

Our main contributions fall into three main areas:

\begin{enumerate}
    \item \emph{Identifying stable functions}: We have found several families of
    stable GDFs.
    \begin{enumerate}
        \item \emph{PC-Lipschitz GDFs}: We showed that Lipschitz continuity with
        respect to the point cloud is a powerful sufficient condition for
        stability. This class of functions is well-behaved and easy to work with.

        \item \emph{Generalised \v{C}ech Filtrations}: We extended known
        stability for the nearest-neighbour distance function to more flexible
        versions, including:
        \begin{enumerate}
            \item Functions of the form $f(P, x) = \min_{p \in P} h(d(x, p))$ for
            monotone Lipschitz $h$.
            \item More general kernels $f(P, x) = \min_{p \in P} h(x, p)$ where $h$
            is Lipschitz in its second argument.
            \item Point cloud dependent kernels $f(P, x) = \min_{p \in P} h(x, p,
            P)$, as long as they satisfy certain Lipschitz conditions.
        \end{enumerate}

        \item \emph{Morse Functions}: For Morse functions, we found that
        stability doesn't require a global property like being PC-Lipschitz.
        Instead, it depends on how the function's critical points behave.
    \end{enumerate}

    \item \emph{Operations on stable functions}: We investigated how stability
    is preserved under common operations:
    
    \begin{enumerate}
        \item For PC-Lipschitz GDFs, stability is preserved by affine transformations,
        addition, and taking the pointwise minimum or maximum.
        \item However, for the broader class of all stable GDFs, we showed this
        isn't always true, although stability is still preserved by affine
        transformations. We provided examples where combining two stable
        functions results in an unstable one.
    \end{enumerate}

    \item \emph{The structure of the function spaces}: We studied the spaces of
    stable and PC-Lipschitz functions, showing that:

    \begin{enumerate}
        \item PC-Lipschitz GDFs form an unbounded distributive lattice under pointwise
        minimum and maximum.
        \item The space of PC-Lipschitz GDFs is convex and closed in various function
        space topologies.
        \item The space of all stable GDFs is path-connected and contractible,
        as well as closed in the topology of pointwise convergence.
        \item Interestingly, the PC-Lipschitz functions are not dense within the
        space of all stable functions, meaning there are stable functions
        that cannot be approximated by these nicer PC-Lipschitz ones.
    \end{enumerate}
\end{enumerate}

Together, these results help build a more solid theoretical foundation for using
GDFs in practice, connecting and extending previous research.

\section{Open questions and future work}

This work also opens up several interesting questions for future research:

\begin{enumerate}
    \item \emph{What makes a function stable?} We found several conditions that
    are sufficient for stability, but we still lack necessary conditions. Fully
    characterizing the space of all stable functions remains an open problem.
    
    \item \emph{How common are PC-Lipschitz functions?}: While the PC-Lipschitz
    condition is useful, how restrictive is it? We could use tools like
    prevalence or shyness to understand how common PC-Lipschitz functions are
    among all stable functions.

    \item \emph{Non-Euclidean spaces}: Our work focused on data in standard
    Euclidean space, but TDA is applicable for other kinds of data as well.
    Extending stability results to those settings is an important next step.
    
    \item \emph{Wasserstein stability}: We measured stability using the
    bottleneck distance. It would be valuable to see how these results change
    when using other metrics between persistence diagrams, like the Wasserstein
    distance.
\end{enumerate}

By mapping out the properties of stable GDFs, this work offers both a practical
guide for researchers applying TDA and a theoretical foundation for those
developing new methods.