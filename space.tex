\chapter{The space of stable functions}
\label{chap:space}

In the previous chapters, we identified several classes of stable generalized
density functions and explored how they can be combined to form new stable
functions.
This naturally leads to considering the collections of these functions as
algebraic structures and topological spaces.
Let $\mathcal{S}_c$ denote the set of all $c$-stable GDFs
and $\mathcal{L}_c$ denote the set of all $c$-PC-Lipschitz GDFs.
We have established that $\mathcal{L}_c \subseteq \mathcal{S}_c$
(Lemma~\ref{lem:pc_lipschitz_stable}) and that this inclusion is proper
(Example~\ref{ex:x_plus_g}).

This chapter investigates the algebraic and topological structure of these sets.
We will first explore $\mathcal{L}_c$, the more structured of the two spaces.
We will then turn our attention to the broader space $\mathcal{S}_c$ and
conclude by examining the relationship between them.

\section{The space of $c$-PC-Lipschitz functions}
\label{sec:space_pc_lipschitz}

The set $\mathcal{L}_c$ exhibits several structural properties.

\begin{theorem}
    The set $\mathcal{L}_c$ forms a distributive lattice with respect to the
    operations of pointwise minimum and pointwise maximum, that is, the minimum
    and maximum of two $c$-stable GDFs is also a $c$-stable GDF and the following laws hold:
    \begin{align}
        \max(f, \min(f, g)) & = f \\
        \min(f, \max(f, g)) & = f \\
        \max(f, \min(g, h)) & = \min(\max(f, g), \max(f, h)) \\
        \min(f, \max(g, h)) & = \max(\min(f, g), \min(f, h)).
    \end{align}
    This lattice is bounded neither from above nor from below.
\end{theorem}
\begin{proof}
    \begin{enumerate}
        \item Both minimum and maximum are closed in $\mathcal{L}_c$ by
            Theorem~\ref{thm:min_pc_lipschitz}. The absorption laws follow from
            the fact that they hold for $(\R, \min, \max)$.
        \item Distributivity follows from the fact that it holds for $(\R, \min, \max)$.
        \item To show that the lattice is not bounded, suppose $f$ is a top
            element. $f + 1$ is also in $\mathcal{L}_c$ by
            Theorem~\ref{thm:constant_addition} and $\max(f, f + 1) = f + 1 \neq
            f$, thus $f$ cannot be a top element. 
        \item We can similarly show that there is no bottom element by
            considering $f - 1$.
    \end{enumerate}
\end{proof}

While $\mathcal{L}_c$ has a lattice structure, it does not form a vector space.
As shown in Theorem~\ref{thm:sum_pc_lipschitz}, the sum of two
functions in $\mathcal{L}_c$ lies in $\mathcal{L}_{2c}$, but not necessarily in
$\mathcal{L}_c$. This lack of closure under addition prevents $\mathcal{L}_c$
from being a vector space. Using an alternative operation such as averaging,
fails to satisfy the axioms for scalar multiplication.

On the other hand, the space of all PC-Lipschitz GDFs
$\mathcal{L} = \bigcup_{c \in \R^+} \mathcal{L}_c$ does form a vector space.
However, this space is of less interest for the purposes of TDA, as practical
applications do not concern themselves with merely stable functions, but rather
with $c$-stable functions for some fixed $c$.
For a comprehensive treatment of general Lipschitz function spaces,
we refer the interested reader to the work of N.~Weaver~\cite{weaver2018lipschitz}.

We now turn to topological properties of $\mathcal{L}_c$.

\begin{theorem}
    The set $\mathcal{L}_c$ is convex.
\end{theorem}
\begin{proof}
    Let $f, g \in \mathcal{L}_c$ and let $\alpha \in [0, 1]$.
    We need to show that the convex combination $\alpha f + (1 - \alpha) g$ is
    also in $\mathcal{L}_c$.

    Since $\alpha \geq 0$, by Theorem~\ref{thm:constant_nonneg_multiplication},
    we have $\alpha f \in \mathcal{L}_{\alpha c}$.
    Similarly, as $(1 - \alpha) \geq 0$, $(1 - \alpha) g \in \mathcal{L}_{(1 - \alpha)c}$.

    Finally, by Theorem~\ref{thm:sum_pc_lipschitz},
    \begin{equation}
        (\alpha f + (1 - \alpha) g) \in \mathcal{L}_{\alpha c + (1 - \alpha)c} = \mathcal{L}_c.
    \end{equation}
\end{proof}

\begin{corollary}
    Let $\mathcal{F}$ be a topological vector space of functions containing
    $\mathcal{L}_c$. Then $\mathcal{L}_c$ is a contractible subset of
    $\mathcal{F}$~\cite{munkres2000topology}.
\end{corollary}
\begin{theorem}
    The set $\mathcal{L}_c$ is closed in the topology of pointwise convergence.
\end{theorem}
\begin{proof}
    Let $\{f^n\}_{n = 1}^\infty$ be a sequence of functions in $\mathcal{L}_c$
    that converges pointwise to a function $f$. We need to show that $f$
    is also in $\mathcal{L}_c$.
    We have:
    \begin{align}
        \norm{f_P - f_Q}_\infty & \leq \norm{f_P - f^n_P}_\infty + \norm{f^n_P - f^n_Q}_\infty + \norm{f^n_Q - f_Q}_\infty \\
        & \leq \norm{f_P - f^n_P}_\infty + c \cdot d_H(P, Q) + \norm{f^n_Q - f_Q}_\infty,
    \end{align}
    which converges to $c \cdot d_H(P, Q)$ and thus $f$ is $c$-PC-Lipschitz.    
\end{proof}
\begin{corollary}
    Since uniform convergence and compact convergence are stronger topologies
    than pointwise convergence, $\mathcal{L}_c$ is also closed in these topologies.
\end{corollary}

\section{The space of $c$-stable functions}

We now turn our attention to the larger space of all $c$-stable GDFs
$\mathcal{S}_c$. As shown in various examples in Chapter~\ref{chap:operations},
$\mathcal{S}_c$ lacks the well-behaved structure of $\mathcal{L}_c$: it is not
closed under addition, pointwise maximum or pointwise minimum.

Despite these limitations, $\mathcal{S}_c$ still possesses important topological
properties.
\begin{theorem}
    The singleton set of the function $g(P, x) = 0$ is a strong deformation
    retract of $\mathcal{S}_c$ in any topology on $\mathcal{F}$ where scalar
    multiplication is continuous.
\end{theorem}
\begin{proof}
    Let $f \in \mathcal{S}_c$.
    We can define a homotopy $H_t(f)$ for $t \in [0, 1]$ as
    \begin{equation}
        H_t(f)(P, x) \coloneqq (1 - t)f(P, x)
    \end{equation}
    Then $H_0(f) = f$ and $H_1(f) = g$.
    As $H_t(g) = g$, this deformation retraction is strong.
\end{proof}
\begin{corollary}
    The set $\mathcal{S}_c \subseteq \mathcal{F}$ is contractible in any
    topology on $\mathcal{F}$ where scalar multiplication is continuous.
\end{corollary}

The path constructed above shows that we can connect any function in
$\mathcal{S}_c$ to the zero function by a continuous path, provided scalar
multiplication is continuous. We can join two such paths to connect any two
functions in $\mathcal{S}_c$:
\begin{theorem}
    The set $\mathcal{S}_c \subseteq \mathcal{F}$ is path-connected in any
    topology on $\mathcal{F}$ where scalar multiplication is continuous.
\end{theorem}
\begin{proof}
    Let $f, g \in \mathcal{S}_c$.
    Let $h_t(P, x)$ be defined for $t \in [0, 1]$ as
    \begin{equation}
        h_t(P, x) \coloneqq \begin{cases}
            (1 - 2t)f(P, x), & t \in [0, 0.5] \\
            (2t - 1)g(P, x), & t \in (0.5, 1]
        \end{cases}
    \end{equation}
    This path starts at $h(0) = f$, passes through the zero function at $h(0.5) = 0$,
    and ends at $h(1) = g$.

    We now verify that $h_t(P, x) \in \mathcal{S}_c$ for all $t \in [0, 1]$.
    For $t \in [0, 0.5]$, the scaling factor is $1 - 2t$, which lies in
    $[0, 1]$, and by Theorem~\ref{thm:constant_nonneg_multiplication},
    $h_t$ is in $\mathcal{S}_c$ for $t \in [0, 0.5]$.
    Similarly, for $t \in (0.5, 1]$, the scaling factor is $2t - 1$, which also
    lies in $[0, 1]$ and thus the whole path $h_t$ is in $\mathcal{S}_c$.
\end{proof}

\begin{theorem}
    The set $\mathcal{S}_c$ is closed in the topology of uniform convergence.
\end{theorem}
\begin{proof}
    Let $\{f^n\}_{n = 1}^\infty$ be a sequence of functions in $\mathcal{S}_c$
    that converges uniformly to a function $f$, that is, $\norm{f - f^n}_\infty \to 0$.
    We need to show that $f$ is also in $\mathcal{S}_c$.

    For any finite sets $P, Q \subseteq \R$, by the triangle inequality, we have:
    \begin{align}
        d_b(\dgm_k(f_P), \dgm_k(f_Q)) & \leq d_b(\dgm_k(f_P), \dgm_k(f^n_P)) +{} \nonumber\\
        & \qquad d_b(\dgm_k(f^n_P), \dgm_k(f^n_Q)) +{} \nonumber\\
        & \qquad d_b(\dgm_k(f^n_Q), \dgm_k(f_Q)) \\
        & \leq \norm{f_P - f^n_P}_\infty + c \cdot d_H(P, Q) + \norm{f^n_Q - f_Q}_\infty,
    \end{align}
    which converges to $c \cdot d_H(P, Q)$ as $n \to \infty$.
    Thus, $f$ is $c$-stable.
\end{proof}

\section{The relationship between $\mathcal{S}_c$ and $\mathcal{L}_c$}

As $\mathcal{L}_c$ is much more well-behaved than $\mathcal{S}_c$,
it would be desirable to have $\mathcal{L}_c$ dense in $\mathcal{S}_c$.
However, as $\mathcal{L}_c$ is closed in the topologies of uniform convergence,
of compact convergence and of pointwise convergence, $\mathcal{L}_c$ is
unfortunately not dense in $\mathcal{S}_c$ in these topologies.

A possible direction for further investigation is to consider the notion of
\emph{shyness}~\cite{ott2005prevalence}, which generalises the concept of
measure zero to infinite-dimensional spaces. This concept allows us to
ask: is the ``usual'' stable function also PC-Lipschitz?

% An alternative approach would be to consider the notion of shyness,
% which is a generalization of the concept of being of measure zero to
% infinite-dimensional spaces. For a thorough introduction to shyness,
% we refer the reader to the work of William Ott and James A.
% Yorke~\cite{ott2005prevalence}.
% \begin{theorem}
%     The set $\mathcal{L}_c$ is shy in $\mathcal{S}_c$ in the topologies of
%     uniform convergence, compact convergence and pointwise convergence.
% \end{theorem}
% \begin{proof}
%     We will show that $\mathcal{L}_c$ has no interior points, i.e. for every
%     $f \in \mathcal{L}_c$ and every neighbourhood $N$ of $f$, there exists a
%     function $g \in N$ such that $g \notin \mathcal{L}_c$, but $g \in \mathcal{S}_c$.
%     This implies that $\mathcal{L}_c$ is shy in $\mathcal{S}_c$~\cite{ott2005prevalence}.
    
%     \todo{Consider arbitrary $f \in \mathcal{L}_c$ and its neighbourhood $N$.
%     Find a function $g \in N$ such that $g \in \mathcal{S}_c \setminus \mathcal{L}_c$
%     by adding a small non-Lipschitz bump to $f$.}
% \end{proof}