\chapter{Background}
\label{chap:background}

This chapter reviews the mathematical foundations for stability of generalized
density functions. We briefly review the core concepts of how real-valued
functions induce filtrations for persistent homology and then focus on
established stability theorems that serve as a foundation and motivation for
this work. We assume the reader has a working knowledge of topological data
analysis, mainly persistent homology. Otherwise, a comprehensive introduction to
the subject is provided in the textbook by Herbert Edelsbrunner and John Harer
\cite{edelsbrunner2010computational}.

At the heart of this thesis lies the concept of a generalized density function (GDF).
\begin{definition}[Generalized density function]
    A generalized density function is a map $f : \mathcal{P}(\R^d) \times \R^d
    \to \R$,
    where $\mathcal{P}(\R^d)$ denotes the collection of all finite subsets of
    $\R^d$. For a given point cloud $P \in \mathcal{P}(\R^d)$, a GDF $f$ induces a
    real-valued function $f_P : \R^d \to \R$ defined by $f_P(x) \coloneqq f(P, x)$.
\end{definition}

These functions $f_P$ are used to construct \emph{sublevel set filtrations}.
\begin{definition}[Sublevel set filtration]
    Given a function $f_P : \R^d \to \R$, its sublevel set at a value $a \in \R$ is
    \begin{equation}
        f_P^{-1}(-\infty, a] = \{ x \in \R^d \mid f_P(x) \leq a \}.
    \end{equation}
    The sublevel set filtration of $g$ is the nested family of sets
    $\{f_P^{-1}(-\infty, a]\}_{a \in \R}$.
\end{definition}
Alternatively, superlevel set filtrations can be used, which are defined by
$\{f_P^{-1}[a, +\infty)\}_{a \in \R}$. The choice between sublevel and superlevel
set filtrations is a matter of convention, as the superlevel set filtration of
$f_P$ is exactly the sublevel set filtration of the function $-f_P$, and vice
versa. For the sake of consistency, we will use sublevel set filtrations
throughout this thesis.

Applying a homology functor $H_k$ to a filtration $\{f_P^{-1}(-\infty, a]\}_{a \in \R}$
yields a persistence module, which tracks the evolution of topological features
as the parameter $a$ varies~\cite{edelsbrunner2010computational}. The
\emph{persistence diagram}, denoted $\dgm_k(f_P)$, is a multiset of points
$(b_i, d_i)$ in the extended plane $\overline{\R}^2$, where $b_i$ represents the
birth time and $d_i$ the death time of a $k$-dimensional feature. An example of
a persistence diagram is shown in Figure~\ref{fig:pd}.

\begin{figure}
    \centering
    \begin{tikzpicture}
        \small
        % Axes
        \draw[->] (0,0) -- (5,0) node[right] {Birth};
        \draw[->] (0,0) -- (0,5.5) node[above] {Death};
        
        % Grid
        % \draw[gray!30] (0,0) grid (6,6);
        
        % Diagonal
        \draw[dashed] (0,0) -- (5,5);
        
        % Point coords
        \coordinate (p1) at (1,3);
        \coordinate (p2) at (2,4);
        \coordinate (p3) at (1.5,2);
        \coordinate (p4) at (3,5);
        
        % Lines to axes
        \draw[dashed,gray] (p1) -- (1,0) node[below,black] {$b_1$};
        \draw[dashed,gray] (p1) -- (0,3) node[left,black] {$d_1$};
        \draw[dashed,gray] (p2) -- (2,0) node[below,black] {$b_2$};
        \draw[dashed,gray] (p2) -- (0,4) node[left,black] {$d_2$};
        \draw[dashed,gray] (p3) -- (1.5,0) node[below,black] {$b_3$};
        \draw[dashed,gray] (p3) -- (0,2) node[left,black] {$d_3$};
        \draw[dashed,gray] (p4) -- (3,0) node[below,black] {$b_4$};
        \draw[dashed,gray] (5,5) -- (0,5) node[left,black] {$\infty$};

        % Points
        \fill (p1) circle (2pt) node[above right] {$p_1$};
        \fill (p2) circle (2pt) node[above right] {$p_2$};
        \fill (p3) circle (2pt) node[above right] {$p_3$};
        \fill (p4) circle (2pt) node[above right] {$p_4$};
    \end{tikzpicture}
    \caption{An example of a persistence diagram. A point $p_i$ is born at $b_i$
    and dies at $d_i$. The point $p_4$ does not die, so its death time is
    $\infty$. The dashed line indicates the diagonal $\Delta$, where $b_i = d_i$.}
    \label{fig:pd}
\end{figure}

\section{Tameness}
\todo{wording in this section}

Many known stability results outlined in
Section~\ref{sec:known_stability_results} as well as the novel results in this
thesis require the involved persistence modules to be \emph{$q$-tame}.
\begin{definition}[$q$-tame persistence module,~\cite{chazal2013structurestabilitypersistencemodules}]
    A persistence module $\mathbb{M}$ with vector spaces $M_a$ and linear maps
    $m_a^b: M_a \to M_b$ is \emph{$q$-tame} if $m_a^b$ is of finite rank for all
    $a < b$.
\end{definition}

$q$-tameness is a generalization of the following notion:
\begin{definition}[Pointwise finite-dimensional persistence module]
    A persistence module $\mathbb{M}$ is \emph{pointwise finite-dimensional
    (PFD)} if for every $a \in \R$, the vector space $M_a$ is
    finite-dimensional.
\end{definition}
\begin{lemma}[\cite{chazal2013structurestabilitypersistencemodules}]
    \label{lem:tameness_pfd}
    If a persistence module is PFD, then it is $q$-tame.
\end{lemma}

The following theorems can be used to prove $q$-tameness of persistence modules
induced by sublevel set filtrations.
\begin{theorem}[\cite{cagliari2010finitenessrankinvariantsmultidimensional}]
    Let $X$ be a realisation of a finite complex, and let $f : X \to \R$ be a
    continuous function. Then the persistence module induced by sublevel sets of
    $f$ is $q$-tame.
\end{theorem}
While this theorem is powerful, it is of limited use in this thesis, as we
consider functions $f_P : \R^d \to \R$, and no triangulation of $\R^d$ is
finite, as $\R^d$ is not compact. This condition is relaxed in the following
theorem:
\begin{theorem}[\cite{chazal2013structurestabilitypersistencemodules}, Corollary 3.34]
    \label{thm:tameness_condition}
    Let $X$ be a realisation of a \emph{locally} finite complex, and let
    $f : X \to \R$ be a continuous function which is bounded below and
    the preimage $f^{-1}(K)$ of every compact set $K \subseteq \R$ is compact.
    Then the persistence module induced by sublevel sets of $f$ is $q$-tame.
\end{theorem}
This theorem is applicable to $\R^d$, as it admits a locally finite
triangulation~\cite{grafakos2008classical}. The condition that the preimage of
every compact set is compact, also known as \emph{properness}, is required to
remove pathological functions such as the sine function, whose sublevel sets
$\sin^{-1}(-\infty, a]$ have infinite-dimensional $0$-homology for all
$a \in (-1, 1)$.

\section{Quantifying stability}

To formalize the notion of stability, we need metrics to compare both point
clouds and persistence diagrams. A common metric for point clouds is the
\emph{Hausdorff distance}, which is also used in the definition of stability of
the nearest-neighbor distance function.
\begin{definition}[Hausdorff distance]
    Given two non-empty sets $P, Q \subseteq \R^d$, where $\R^d$ is a metric
    space, the Hausdorff distance between $P$ and $Q$ is defined as
    \begin{equation}
        d_H(P, Q) = \max\left\{ \sup_{p \in P} \inf_{q \in Q} d(p, q), \sup_{q \in Q} \inf_{p \in P} d(q, p) \right\}.
    \end{equation}
\end{definition}
Intuitively, the Hausdorff distance measures the ``maximum mismatch'' between the two
sets, capturing the largest distance between a point in one set and its closest
neighbor in the other set.
% It should be noted that for arbitrary $P, Q \subseteq X$, the
% Hausdorff distance may be infinite, which makes it not a metric. However, if we
% assume that $P$ and $Q$ are not empty and compact, then the Hausdorff distance is
% a metric.
% \todo{reword for flow}
% This assumption is reasonable in the context of TDA, where we typically work with
% finite non-empty point clouds.

\begin{definition}[Bottleneck distance]
    Given two persistence diagrams $D_1$ and $D_2$, the bottleneck distance
    between them is defined as
    \begin{equation}
        d_b(D_1, D_2) = \inf_\pi \sup_{(b, d) \in D_1} \norm{(b, d) - \pi(b, d)}_\infty,
    \end{equation}
    where the infimum is taken over all bijections $\pi$ between $D_1$ and
    $D_2$, allowing for the possibility of matching points to the diagonal $\Delta = \{(x, x) \mid x \in \overline{\R}\}$.
\end{definition}
The bottleneck distance measures the cost of the ``least expensive'' matching
between the points of the two diagrams, where the cost is defined as the longest
edge in the matching.

% A generalization of the bottleneck distance is the \emph{Wasserstein distance},
% which defines the cost of matching points in terms of the $p$-norm:
% \begin{definition}[Wasserstein distance]
%     Given two persistence diagrams $D_1$ and $D_2$, the Wasserstein distance
%     between them is defined as
%     \begin{equation}
%         d_p(D_1, D_2) = \inf_\pi \left( \sum_{(b, d) \in D_1} \norm{(b, d) - \pi(b, d)}^p_\infty \right)^{1/p},
%     \end{equation}
%     where the infimum is taken over the same bijections $\pi$ as in the
%     bottleneck distance.
% \end{definition}
% The case $p = \infty$ corresponds to the bottleneck distance.
% \todo{possibly remove, if Wasserstein distance isn't used later}

With these metrics defined, we can formally define stability for a GDF.
\begin{definition}
    A generalized density function $f : \mathcal{P}(\R^d) \times \R^d \to \R$ is
    $c$-\emph{stable} if for all non-empty finite point clouds $P, Q \subseteq
    \R^d$ and all homology dimensions $k \geq 0$, we have
    \begin{equation}
        d_b(\dgm_k(f_P), \dgm_k(f_Q)) \leq c \cdot d_H(P, Q).
    \end{equation}
    If no such finite $c$ exists, we say that $f$ is \emph{unstable} or
    $\infty$-stable.
\end{definition}

\section{Known stability results}
\label{sec:known_stability_results}

This section reviews the known stability results for GDFs and simplicial
filtrations that can be represented as GDFs.

\subsection{\v{C}ech filtration}
As mentioned in Chapter~\ref{chap:introduction}, the \v{C}ech filtration can be
represented as the sublevel set filtration of the nearest-neighbor distance
function $f_{\mathrm{dist}}(P, x) \coloneqq \min_{p\in P} d(x, p)$.
The sublevel sets $f_P^{-1}(-\infty, a]$ are precisely
$\bigcup_{p\in P} B(p, a)$, the union of closed balls of radius $a$ centered at
the points of $P$. The nerve of this union of balls defines the \v{C}ech
complex. By the Nerve theorem~\cite{Borsuk1948,leray1945forme}, if $P$ is finite,
the \v{C}ech complex is homotopy equivalent to the union of balls.

The stability of this construction is a well-known result:
\begin{theorem}[Stability of the \v{C}ech filtration, \cite{chazal2013persistencestabilitygeometriccomplexes}]
    Let $f_P(x) \coloneqq \min_{p\in P} d(x, p)$. Then for all finite
    $P, Q \subseteq \mathcal{P}(\R^d)$ and any dimension $k \geq 0$, we have
    \begin{equation}
        d_b(\dgm_k(f_P), \dgm_k(f_Q)) \leq d_H(P, Q).
    \end{equation}
\end{theorem}
Using the terminology of this thesis, we say the nearest-neighbor distance
function is 1-stable.

\subsection{Weighted \v{C}ech filtration}

The weighted \v{C}ech filtration is a generalization of the \v{C}ech filtration
used to define the DTM-filtration~\cite{anai2020dtm}.

\begin{definition}[Weighted \v{C}ech filtration]
    Let $q \in [1, \infty)$, $P \subseteq \R^d$ and
    $g : P \to \R_{\geq 0}$. For every $p \in P, a \in \R^+$, we define the
    function $r_p(a)$ to be:
    \begin{equation}
        r_p^{(q)}(a) = \begin{cases}
            - \infty, & \text{if } a < g(p), \\
            (a^q - g(p)^q)^{1/q}, & \text{otherwise}.
        \end{cases}
    \end{equation}
    For $q = \infty$, we also define
    \begin{equation}
        r_p^{(q)}(a) = \begin{cases}
            - \infty, & \text{if } a < g(p), \\
            a, & \text{otherwise}.
        \end{cases}
    \end{equation}
    $g(p)$ acts as a weight for each point $p \in P$, influencing the radius
    $r_p^{(q)}(a)$ of the ball $B(p, r_p^{(q)}(a))$. Concretely, the weighted
    \v{C}ech filtration is defined as
    \begin{equation}
        V^a_q[P, g] = \bigcup_{p \in P} B(p, r_p^{(q)}(a)),
    \end{equation}
    where the balls are closed, and the ball of radius $-\infty$ is the empty
    set.
\end{definition}
The weighted \v{C}ech filtration is a strict generalization of the \v{C}ech
filtration, as the \v{C}ech filtration is the special case where $g(p) = 0$ for
all $p \in P$.

Similarly to the \v{C}ech filtration, this filtration can be represented as a
GDF of the form
\begin{equation}
    f_{g, q}(P, x) = \min_{p \in P} (d(x, p)^{q} + g(p)^{q})^{1/q},
\end{equation}
where similarly to the $L_\infty$-norm, when $q = \infty$, the term
$(d(x, p)^q + g(p)^q)^{1/q}$ is replaced by $\max(d(x, p), g(p))$.
A proof of the equivalence of this GDF and the weighted \v{C}ech filtration is
given in Appendix~\ref{app:weighted_cech}.

If $g$ is Lipschitz continuous, then the weighted \v{C}ech filtration is
guaranteed to be stable:
\begin{theorem}[Stability of the weighted \v{C}ech filtration, \cite{anai2020dtm}]
    \label{thm:weighted_cech_stability}
    Let $P, Q \subseteq \R^d$ be compact and $g : P \cup Q \to \R^+$ be a
    Lipschitz continuous function with Lipschitz constant $c$. Then for any
    dimension $k \geq 0$, we have
    \begin{equation}
        d_b(\dgm_k(V^a_q[P, g]), \dgm_k(V^a_q[Q, g])) \leq (1 + c^q)^{1/q} \cdot d_H(P, Q).
    \end{equation}
\end{theorem}
This theorem immediately implies that the corresponding GDF
is $(1 + c^q)^{1/q}$-stable.

\subsection{Stability of $L_\infty$-norm-bounded functions}

This fundamental result relates the bottleneck distance between persistence
diagrams of two functions to the $L_\infty$-distance between the functions
themselves.

\begin{theorem}[Stability of persistence diagrams, \cite{cohen2005stability}]
    \label{thm:stability_sup_norm}
    Let $X$ be a triangulable space, and $g, h : X \to \R$ be two continuous
    tame functions. Then for all $k \geq 0$, the persistence diagrams of $g$ and
    $h$ satisfy
    \begin{equation}
        d_b(\dgm_k(g), \dgm_k(h)) \leq \norm{g - h}_{\infty}.
    \end{equation}
\end{theorem}
\begin{remark}
    This theorem also holds when $g$ and $h$ are not necessairly tame, as long
    as the resulting persistence modules are
    $q$-tame~\cite{chazal2013structurestabilitypersistencemodules}.
\end{remark}

Importantly, this theorem is applicable to the case $X = \R^d$, as the space
$\R^d$ is triangulable~\cite{moise2013geometric}. We can easily adapt this
theorem to the case of GDFs:
\begin{equation}
    d_b(\dgm_k(f_P), \dgm_k(f_Q)) \leq \norm{f_P - f_Q}_{\infty}.
\end{equation}
This theorem directly implies that if a GDF $f$ satisfies
$\norm{f_P - f_Q}_\infty \leq c \cdot d_H(P, Q)$ for some constant $c$, then
$f$ is $c$-stable. This provides a powerful tool for proving stability of GDFs.
As an example, we prove 1-stability of nearest neighbor distance function using
this theorem. To do so, we first need to show the following lemma:
\begin{lemma}
    \label{lem:cech_tame}
    The persistence modules induced by the nearest-neighbor distance function
    $f_{\mathrm{dist}}(P, x)$ are $q$-tame.
\end{lemma}
\begin{proof}
    The \v{C}ech filtration $\check{C}(P)$ is a filtration of a finite
    simplicial complex if $P$ is finite, and its persistence modules
    $H_k(\check{C}(P))$ are thus pointwise finite-dimensional. The sublevel set
    filtration induced by $f_{\mathrm{dist}}$ is homotopy equivalent to
    $\check{C}(P)$~\cite{schnider2024introduction}, which implies that the
    persistence modules corresponding to $f_{\mathrm{dist}}$ are also PFD, which
    implies $q$-tameness by Lemma~\ref{lem:tameness_pfd}.
\end{proof}
Now we can prove the stability of the nearest-neighbor distance function:
\begin{proof}
    Let $P, Q \subseteq \R^d$ be two finite point clouds with Hausdorff distance
    $d = d_H(P, Q)$. Let $x$ be an arbitrary point in $\R^d$ with the nearest
    point in $P$ denoted by $p$. This implies that $f_{\mathrm{dist}, P}(x) =
    d(x, p)$.  By the definition of the Hausdorff distance, there exists a point
    $q' \in Q$ such that $d(p, q') \leq d$. By the triangle inequality, we have
    \begin{equation}
        d(x, q') \leq d(x, p) + d(p, q') \leq d(x, p) + d.
    \end{equation}
    and thus:
    \begin{equation}
        f_{\mathrm{dist}, Q}(x) = \min_{q \in Q} d(x, q) \leq d(x, q') \leq d(x, p) + d = f_{\mathrm{dist}, P}(x) + d.
    \end{equation}
    We can show that $f_{\mathrm{dist}, P}(x) \leq f_{\mathrm{dist}, Q}(x) + d$
    by the same reasoning. Thus, we have
    \begin{equation}
        \norm{f_{\mathrm{dist}, P} - f_{\mathrm{dist}, Q}}_\infty \leq d,
    \end{equation}
    which implies that $f_{\mathrm{dist}}$ is 1-stable by
    Theorem~\ref{thm:stability_sup_norm} and Lemma~\ref{lem:cech_tame}.
\end{proof}

\subsection{Stability of functions with interleaved persistence modules}

An alternative approach to show stability is to use the \emph{interleaving
distance}, which bounds the bottleneck distance not only from above, but also
from below.
\begin{definition}[Interleaving of persistence modules]
    \label{def:interleaving}
    Two persistence modules $\mathbb{M}$ and $\mathbb{N}$ with $M_a, N_a$ as
    their respective vector spaces and $m_a^{a'}, n_a^{a'}$ as their
    homomorphisms are \emph{$\varepsilon$-interleaved}
    for $\varepsilon \geq 0$ if there exist families of maps $\varphi_a : M_a
    \to N_{a + \varepsilon}$ and $\psi_a : N_a \to N_{a + \varepsilon}$
    such that the following diagrams commute for all $a < a'$:
    \begin{equation*}
        \begin{tikzcd}
            M_a \arrow{rr}{m_a^{a'}} \arrow{dr}[swap]{\varphi_a} & & M_{a'} \arrow{dr}{\varphi_{a'}} \\
            & N_{a + \varepsilon} \arrow{rr}[swap]{n_{a + \varepsilon}^{a' + \varepsilon}} & & N_{a' + \varepsilon}
        \end{tikzcd}
        \begin{tikzcd}
            M_a \arrow{dr}[swap]{\varphi_a} \arrow{rr}{m_a^{a + 2\varepsilon}} && M_{a + 2\varepsilon} \\
            & N_{a + \varepsilon} \arrow{ur}[swap]{\psi_{a + \varepsilon}}
        \end{tikzcd}
    \end{equation*}
    \begin{equation*}
        \begin{tikzcd}
            & M_{a + \varepsilon} \arrow{rr}{n_{a + \varepsilon}^{a' + \varepsilon}} & & M_{a' + \varepsilon} \\
            N_a \arrow{rr}[swap]{n_a^{a'}} \arrow{ur}{\psi_a} & & N_{a'} \arrow{ur}[swap]{\psi_{a'}}
        \end{tikzcd}
        \begin{tikzcd}
            & M_{a + \varepsilon} \arrow{dr}{\varphi_{a + \varepsilon}} \\
            N_a \arrow{ur}{\psi_a} \arrow{rr}[swap]{n_a^{a + 2\varepsilon}} && N_{a + 2\varepsilon}
        \end{tikzcd}
    \end{equation*}
\end{definition}
Note that $\mathbb{M}$ and $\mathbb{N}$ are $0$-interleaved if and only if
they are isomorphic. The triangular diagrams on the right collapse to diagrams
that show that $\varphi_t$ and $\psi_t$ are inverses of each other, while
the trapezoidal diagrams on the left ensure they commute with the linear maps
$m$ and $n$.

\begin{definition}[Interleaving distance]
    The interleaving distance between two persistence modules $\mathbb{M}$ and
    $\mathbb{N}$ is given by:
    \begin{equation}
        d_i(\mathbb{M}, \mathbb{N}) = \inf \{\varepsilon \mid \text{$\mathbb{M}$ and $\mathbb{N}$ are $\varepsilon$-interleaved}\}.
    \end{equation}
\end{definition}

Finally, the interleaving distance and the bottleneck distance are equal under a
mild tameness condition:
\begin{theorem}[Isometry theorem,~\cite{chazal2013structurestabilitypersistencemodules}]
    \label{thm:isometry}
    Let $\mathbb{M}, \mathbb{N}$ be $q$-tame persistence modules. Then
    \begin{equation}
        d_i(\mathbb{M}, \mathbb{N}) = d_b(\dgm(\mathbb{M}), \dgm(\mathbb{N})).
    \end{equation}
\end{theorem}

Additionally, we can define interleaving for filtrations:
\begin{definition}[Interleaving of filtrations]
    Let $\mathcal{F}, \mathcal{G}$ be filtrations over $\R$, that is,
    $\R$-indexed sequence of nested subspaces of some topological space $X$.
    $\mathcal{F}$ and $\mathcal{G}$ are \emph{$\varepsilon$-interleaved}
    if there exist families of maps $\varphi_a : F_a \to G_{a + \varepsilon}$
    and $\psi_a : G_a \to F_{a + \varepsilon}$ such that the same diagrams as in
    Definition~\ref{def:interleaving} commute up to homotopy.
\end{definition}
The interleaving distance between two filtrations is defined the same way as for
persistence modules. The following lemma shows that the former bounds the latter
from above:
\begin{lemma}[\cite{schnider2024introduction}]
    \label{lem:interleaving_distance}
    For any two filtrations $\mathcal{F}$ and $\mathcal{G}$ over $\R$, we
    have
    \begin{equation}
        d_i(H_p \mathcal{F}, H_p \mathcal{G}) \leq d_i(\mathcal{F}, \mathcal{G}).
    \end{equation}
\end{lemma}

With these results in hand, we can apply the interleaving approach to once again
prove 1-stability of the \v{C}ech filtration.
\begin{proof}[\cite{schnider2024introduction}]
    Let $X$ be a topological space, and $P, Q \subseteq X$ be two finite
    point clouds with Hausdorff distance $d = d_H(P, Q)$.
    Recall that the \v{C}ech complex $\check{C}^r(P)$ is homotopy equivalent
    to the union of the balls $\bigcup_{p \in P} B(p, r)$.
    Let $x \in B(p, r)$. By the definition of the Hausdorff distance, there exists
    a point $q \in Q$ such that $d(p, q) \leq d$. Therefore, by the triangle
    inequality we have
    \begin{equation}
        d(x, q) \leq d(x, p) + d(p, q) \leq r + d,
    \end{equation}
    which implies that $x \in B(q, r + d)$. This provides an inclusion
    \begin{equation}
        \bigcup_{p \in P} B(p, r) \subseteq \bigcup_{q \in Q} B(q, r + d).
    \end{equation}
    With these two facts, we see that the following diagram commutes up to
    homotopy, as all maps are either inclusions or homotopies:
    \begin{equation}
        \begin{tikzcd}
            \check{C}^r(P) \arrow[hook]{r} \arrow[leftrightarrow]{d}{\simeq}
            & \check{C}^{r + d}(P) \arrow[hook]{r} \arrow[leftrightarrow]{d}{\simeq}
            & \check{C}^{r + 2d}(P) \arrow[leftrightarrow]{d}{\simeq} \\
            \bigcup_{p \in P} B(p, r) \arrow[hook]{r} \arrow[hook]{dr}
            & \bigcup_{p \in P} B(p, r + d) \arrow[hook]{r} \arrow[hook]{dr}
            & \bigcup_{p \in P} B(p, r + 2d) \\
            \bigcup_{q \in Q} B(q, r) \arrow[hook]{r} \arrow[hook]{ur}
            & \bigcup_{q \in Q} B(q, r + d) \arrow[hook]{r} \arrow[hook]{ur}
            & \bigcup_{q \in Q} B(q, r + 2d) \\
            \check{C}^r(Q) \arrow[hook]{r} \arrow[leftrightarrow]{u}{\simeq}
            & \check{C}^{r + d}(Q) \arrow[hook]{r} \arrow[leftrightarrow]{u}{\simeq}
            & \check{C}^{r + 2d}(Q) \arrow[leftrightarrow]{u}{\simeq}
        \end{tikzcd}
    \end{equation}
    All diagrams that are needed to show that $\check{C}^r$ and $\check{C}^r(Q)$ are
    $d$-interleaved are included in this diagram, and thus we have
    \begin{equation}
        d_i(\check{C}(P), \check{C}(Q)) \leq d.
    \end{equation}
    By Lemma~\ref{lem:cech_tame} and Theorem~\ref{thm:isometry}, as well as
    Lemma~\ref{lem:interleaving_distance}, we have
    \begin{align}
        d_b(\dgm_k(\check{C}(P)), \dgm_k(\check{C}(Q)))
        & \leq d_i(H_k(\check{C}(P)), H_k(\check{C}(Q))) \\
        & \leq d_i(\check{C}(P), \check{C}(Q)) \\
        & \leq d.
    \end{align}
\end{proof}