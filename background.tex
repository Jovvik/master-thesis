\chapter{Background}
\label{chap:background}

This chapter reviews the mathematical foundations for stability of generalized
density functions. We briefly review the core concepts of how real-valued
functions induce filtrations for persistent homology and then focus on
established stability theorems that serve as a foundation and motivation for
this work. We assume the reader has a working knowledge of topological data
analysis, mainly persistent homology.

\section{Generalized density functions and filtrations}

At the heart of this thesis lies the concept of a generalized density function (GDF).
\begin{definition}[Generalized density function]
    A generalized density function is a map $f : \mathcal{P}(X) \times X \to \R$,
    where $\mathcal{P}(X)$ denotes the power set of $X$. For a given point cloud
    $P \in \mathcal{P}(X)$, a GDF $f$ induces a real-valued function $f_P : X
    \to \R$ defined by $f_P(x) \coloneqq f(P, x)$.
\end{definition}

These functions $f_P$ are used to construct \emph{sublevel set filtrations}.
\begin{definition}[Sublevel set filtration]
    Given a function $f_P : X \to \R$, its sublevel set at a value $a \in \R$ is
    \begin{equation}
        f_P^{-1}(-\infty, a] = \{ x \in X \mid f_P(x) \leq a \}.
    \end{equation}
    The sublevel set filtration of $g$ is the nested family of sets
    $\{f_P^{-1}(-\infty, a]\}_{a \in \R}$.
\end{definition}
Alternatively, superlevel set filtrations can be used, which are defined by
$\{f_P^{-1}[a, +\infty)\}_{a \in \R}$. The choice between sublevel and superlevel
set filtrations is a matter of convention, as the superlevel set filtration of
$f_P$ is exactly the sublevel set filtration of the function $-f_P$, and vice
versa. For the sake of consistency, we will use sublevel set filtrations
throughout this thesis.

Applying a homology functor $H_k$ to the filtration $\{f_P^{-1}(-\infty, a]\}_{a \in \R}$
yields a persistence module, which tracks the evolution of topological features
as the parameter $a$ varies. The \emph{persistence diagram}, denoted $\dgm_k(f_P)$,
is a multiset of points $(b_i, d_i)$ in the extended plane $\overline{\R}^2$,
where $b_i$ represents the birth time and $d_i$ the death time of a
$k$-dimensional feature. An example of a persistence diagram is shown in
Figure~\ref{fig:pd}.

\begin{figure}
    \centering
    \begin{tikzpicture}
        \small
        % Axes
        \draw[->] (0,0) -- (5,0) node[right] {Birth};
        \draw[->] (0,0) -- (0,5.5) node[above] {Death};
        
        % Grid
        % \draw[gray!30] (0,0) grid (6,6);
        
        % Diagonal
        \draw[dashed] (0,0) -- (5,5);
        
        % Point coords
        \coordinate (p1) at (1,3);
        \coordinate (p2) at (2,4);
        \coordinate (p3) at (1.5,2);
        \coordinate (p4) at (3,5);
        
        % Lines to axes
        \draw[dashed,gray] (p1) -- (1,0) node[below,black] {$b_1$};
        \draw[dashed,gray] (p1) -- (0,3) node[left,black] {$d_1$};
        \draw[dashed,gray] (p2) -- (2,0) node[below,black] {$b_2$};
        \draw[dashed,gray] (p2) -- (0,4) node[left,black] {$d_2$};
        \draw[dashed,gray] (p3) -- (1.5,0) node[below,black] {$b_3$};
        \draw[dashed,gray] (p3) -- (0,2) node[left,black] {$d_3$};
        \draw[dashed,gray] (p4) -- (3,0) node[below,black] {$b_4$};
        \draw[dashed,gray] (5,5) -- (0,5) node[left,black] {$\infty$};

        % Points
        \fill (p1) circle (2pt) node[above right] {$p_1$};
        \fill (p2) circle (2pt) node[above right] {$p_2$};
        \fill (p3) circle (2pt) node[above right] {$p_3$};
        \fill (p4) circle (2pt) node[above right] {$p_4$};
    \end{tikzpicture}
    \caption{An example of a persistent diagram. A point $p_i$ is born at $b_i$ and dies at $d_i$. The point $p_4$ does not die, so its death time is
    $\infty$. The dashed line indicates the diagonal $\Delta$, where $b_i = d_i$.}
    \label{fig:pd}
\end{figure}

\section{Quantifying stability}

To formalize the notion of stability, we need metrics to compare both point
clouds and persistence diagrams. A common metric for point clouds is the
\emph{Hausdorff distance}, which is also used in the definition of stability of
the nearest-neighbor distance function.
\begin{definition}[Hausdorff distance]
    Given two non-empty sets $P, Q \subseteq X$, where $X$ is a metric space,
    the Hausdorff distance between $P$ and $Q$ is defined as
    \begin{equation}
        d_H(P, Q) = \max\left\{ \sup_{p \in P} \inf_{q \in Q} d(p, q), \sup_{q \in Q} \inf_{p \in P} d(q, p) \right\}.
    \end{equation}
\end{definition}
Intuitively, the Hausdorff distance measures the ``maximum mismatch'' between the two
sets, capturing the largest distance between a point in one set and its closest
neighbor in the other set.
% It should be noted that for arbitrary $P, Q \subseteq X$, the
% Hausdorff distance may be infinite, which makes it not a metric. However, if we
% assume that $P$ and $Q$ are not empty and compact, then the Hausdorff distance is
% a metric.
% \todo{reword for flow}
% This assumption is reasonable in the context of TDA, where we typically work with
% finite non-empty point clouds.

As the Hausdorff distance requires $X$ to be a metric space, we will assume this
throughout the thesis.

\begin{definition}[Bottleneck distance]
    Given two persistence diagrams $D_1$ and $D_2$, the bottleneck distance
    between them is defined as
    \begin{equation}
        d_b(D_1, D_2) = \inf_\pi \sup_{(b, d) \in D_1} \norm{(b, d) - \pi(b, d)}_\infty,
    \end{equation}
    where the infimum is taken over all bijections $\pi$ between $D_1$ and
    $D_2$, allowing for the possibility of matching points to the diagonal $\Delta = \{(x, x) \mid x \in \overline{\R}\}$.
\end{definition}
The bottleneck distance measures the cost of the ``most expensive'' matching
between the points of the two diagrams, where the cost is defined as the longest
edge in the matching.

% A generalization of the bottleneck distance is the \emph{Wasserstein distance},
% which defines the cost of matching points in terms of the $p$-norm:
% \begin{definition}[Wasserstein distance]
%     Given two persistence diagrams $D_1$ and $D_2$, the Wasserstein distance
%     between them is defined as
%     \begin{equation}
%         d_p(D_1, D_2) = \inf_\pi \left( \sum_{(b, d) \in D_1} \norm{(b, d) - \pi(b, d)}^p_\infty \right)^{1/p},
%     \end{equation}
%     where the infimum is taken over the same bijections $\pi$ as in the
%     bottleneck distance.
% \end{definition}
% The case $p = \infty$ corresponds to the bottleneck distance.
% \todo{possibly remove, if Wasserstein distance isn't used later}

With these metrics, we can formally define stability for a GDF.
\begin{definition}
    A generalized density function $f : \mathcal{P}(X) \times X \to \R$ is
    $c$-\emph{stable} if for all non-empty finite point clouds $P, Q \subseteq
    X$ and all homology dimensions $k \geq 0$, we have
    \begin{equation}
        d_b(\dgm_k(f_P), \dgm_k(f_Q)) \leq c \cdot d_H(P, Q).
    \end{equation}
    If no such finite $c$ exists, we say that $f$ is \emph{unstable} or
    $\infty$-stable.
\end{definition}

\section{Known stability results}

This section reviews the known stability results for GDFs and simplicial
filtrations that can be represented as GDFs.

\subsection{\v{C}ech filtration}
As mentioned in Chapter~\ref{chap:introduction}, the \v{C}ech filtration can be
represented as the sublevel set filtration of the nearest-neighbor distance
function $f_{\mathrm{dist}}(P, x) \coloneqq \min_{p\in P} d(x, p)$.
The sublevel sets $f_{\mathrm{dist}, P}^{-1}(-\infty, a]$ are precisely
$\bigcup_{p\in P} B(p, a)$, the union of closed balls of radius $a$ centered at
the points of $P$. The nerve of this union of balls defines the \v{C}ech
complex. By the Nerve theorem~\cite{Borsuk1948,leray1945forme}, if $P$ is finite,
the \v{C}ech complex is homotopy equivalent to the union of balls.

The stability of this construction is a well-known result:
\begin{theorem}[Stability of the \v{C}ech filtration, \cite{chazal2013persistencestabilitygeometriccomplexes}]
    Let $f_{\mathrm{dist}, P}(x) \coloneqq \min_{p\in P} d(x, p)$. Then for all finite
    $P, Q \subseteq \mathcal{P}(X)$ and any dimension $k \geq 0$, we have
    \begin{equation}
        d_b(\dgm_k(f_{\mathrm{dist}, P}), \dgm_k(f_{\mathrm{dist}, Q})) \leq d_H(P, Q).
    \end{equation}
\end{theorem}
Using the terminology of this thesis, we say the nearest-neighbor distance
function is 1-stable.

\subsection{Weighted \v{C}ech filtration}

The weighted \v{C}ech filtration is a generalization of the \v{C}ech filtration
used to define the DTM-filtration~\cite{anai2020dtm}.

\begin{definition}[Weighted \v{C}ech filtration]
    Let $X = \R^d$, $q \in [1, \infty)$, $P \subseteq \R^d$ and
    $g : P \to \R_{\geq 0}$. For every $p \in P, a \in \R^+$, we define the
    function $r_p(a)$ to be:
    \begin{equation}
        r_p^{(q)}(a) = \begin{cases}
            - \infty, & \text{if } a < g(p), \\
            (a^q - g(p)^q)^{1/q}, & \text{otherwise}.
        \end{cases}
    \end{equation}
    For $q = \infty$, we also define
    \begin{equation}
        r_p^{(q)}(a) = \begin{cases}
            - \infty, & \text{if } a < g(p), \\
            a, & \text{otherwise}.
        \end{cases}
    \end{equation}
    $g(p)$ acts as a weight for each point $p \in P$, influencing the radius
    $r_p^{(q)}(a)$ of the ball $B(p, r_p^{(q)}(a))$. Concretely, the weighted
    \v{C}ech filtration is defined as
    \begin{equation}
        V^a_q[P, g] = \bigcup_{p \in P} B(p, r_p^{(q)}(a)),
    \end{equation}
    where the balls are closed, and the ball of radius $-\infty$ is the empty
    set.
\end{definition}
The weighted \v{C}ech filtration is a strict generalization of the \v{C}ech
filtration, as the \v{C}ech filtration is the special case where $g(p) = 0$ for
all $p \in P$.

Similarly to the \v{C}ech filtration, this filtration can be represented as a
GDF of the form
\begin{equation}
    f_{g, q}(P, x) = \min_{p \in P} (d(x, p)^{q} + g(p)^{q})^{1/q},
\end{equation}
where similarly to the $L_\infty$-norm, when $q = \infty$, the term
$(d(x, p)^q + g(p)^q)^{1/q}$ is replaced by $\max(d(x, p), g(p))$.
A proof of the equivalence of this GDF and the weighted \v{C}ech filtration is
given in Appendix~\ref{app:weighted_cech}.

If $g$ is Lipschitz continuous, then the weighted \v{C}ech filtration is stable:
\begin{theorem}[Stability of the weighted \v{C}ech filtration, \cite{anai2020dtm}]
    Let $P, Q \subset \R^n$ be compact and $g : P \cup Q \to \R^+$ be a
    Lipschitz continuous function with Lipschitz constant $c$. Then for any
    dimension $k \geq 0$, we have
    \begin{equation}
        d_b(\dgm_k(V^a_q[P, g]), \dgm_k(V^a_q[Q, g])) \leq (1 + c^q)^{1/q} \cdot d_H(P, Q),
    \end{equation}
\end{theorem}
This theorem immediately implies that the corresponding GDF
is $(1 + c^q)^{1/q}$-stable.

\subsection{Stability of $L_\infty$-norm-bounded functions}

This fundamental result relates the bottleneck distance between persistence
diagrams of two functions to the $L_\infty$-distance between the functions
themselves.

\begin{theorem}[Stability of persistence diagrams, \cite{cohen2005stability}]
    Let $X$ be a triangulable space, and $g, h : X \to \R$ be two continuous
    tame functions. Then for all $k \in \N$, the persistence diagrams of $g$ and
    $h$ satisfy
    \begin{equation}
        d_b(\dgm_k(g), \dgm_k(h)) \leq \norm{g - h}_{\infty}.
    \end{equation}
\end{theorem}
This theorem can be immediately adapted to the case of GDFs:
\begin{equation}
    d_b(\dgm_k(f_P), \dgm_k(f_Q)) \leq \norm{f_P - f_Q}_{\infty}.
\end{equation}
This theorem directly implies that if a GDF $f$ satisfies
$\norm{f_P - f_Q}_\infty \leq c \cdot d_H(P, Q)$ for some constant $c$, then
$f$ is $c$-stable. This provides a powerful tool for proving stability of GDFs.
