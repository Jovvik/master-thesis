\chapter{Background}

This chapter introduces the mathematical concepts and tools used within this
thesis, as well as known results. We begin with \emph{persistent homology}, a
fundamental tool for measuring topological features at different scales.
\todo{finish}

\section{Persistent homology}

\emph{Simplicial complexes} are a combinatorial structure used to study topological
spaces. They are built from simplices, which are the generalization of
points, line segments, triangles, and higher-dimensional shapes. A simplicial
complex is a set of simplices that satisfy certain conditions \todo{say which ones}.
A \emph{filtered} simplicial complex is a collection of simplices
$(K_\varepsilon)_{\varepsilon \in \R}$ that is indexed by a parameter $\varepsilon$,
where $K_\varepsilon \subseteq K_{\varepsilon'}$ for $\varepsilon < \varepsilon'$.

A real-valued function $f: X \to \R$ on a topological space $X$ induces a
filtered simplicial complex by considering the sublevel sets
\todo{shit definition, talk about triangulation}
\[
    K_\varepsilon = \{ x \in X \mid f(x) \leq \varepsilon \}.
\]
Such a filtration is called a \emph{sublevelset filtration}.

\begin{figure}
    \centering
    \begin{tikzpicture}
        \small
        % Axes
        \draw[->] (0,0) -- (6,0) node[right] {Birth};
        \draw[->] (0,0) -- (0,6) node[above] {Death};
        
        % Grid
        % \draw[gray!30] (0,0) grid (6,6);
        
        % Diagonal
        \draw[dashed] (0,0) -- (6,6);
        
        % Point coords
        \coordinate (p1) at (1,3);
        \coordinate (p2) at (2,4);
        \coordinate (p3) at (3,5);
        \coordinate (p4) at (1.5,2);
        
        % Lines to axes
        \draw[dashed,gray] (p1) -- (1,0) node[below,black] {$b_1$};
        \draw[dashed,gray] (p1) -- (0,3) node[left,black] {$d_1$};
        \draw[dashed,gray] (p2) -- (2,0) node[below,black] {$b_2$};
        \draw[dashed,gray] (p2) -- (0,4) node[left,black] {$d_2$};
        \draw[dashed,gray] (p3) -- (3,0) node[below,black] {$b_3$};
        \draw[dashed,gray] (p3) -- (0,5) node[left,black] {$d_3$};
        \draw[dashed,gray] (p4) -- (1.5,0) node[below,black] {$b_4$};
        \draw[dashed,gray] (p4) -- (0,2) node[left,black] {$d_4$};

        % Points
        \fill (p1) circle (2pt) node[above right] {$p_1$};
        \fill (p2) circle (2pt) node[above right] {$p_2$};
        \fill (p3) circle (2pt) node[above right] {$p_3$};
        \fill (p4) circle (2pt) node[above right] {$p_4$};
    \end{tikzpicture}
    \caption{An example of a persistent diagram. A point $p_i$ is born at $b_i$ and dies at $d_i$.}
    \label{fig:pd}
\end{figure}

A filtered simplicial complex can be summarized by its \emph{persistent homology},
which serves as a fundamental tool in TDA that tracks how topological features
(connected components, voids) appear and disappear as a parameter is
varied~\cite{edelsbrunner2010computational}. Given a filtered
simplicial complex $(K_\epsilon)_{\epsilon \in \R}$, persistent homology
computes pairs $(b_i, d_j)$ where a feature appears at
$\epsilon = b_i$ (birth) and disappears at $\epsilon = d_i$ (death).

These birth--death pairs can be visualized in a \emph{persistence diagram},
where each pair is shown as a point in the Euclidean plane. The distance of a
point from the diagonal indicates the feature's persistence, which is often
interpreted as a measure of its significance. Figure~\ref{fig:pd} shows an
example of a persistence diagram.

Two persistence diagrams can be compared using the \emph{bottleneck distance},
which measures the cost of transforming one diagram into another by
matching points. Formally, given two persistence diagrams $D_1$ and $D_2$, the
bottleneck distance is defined as
\begin{equation}
    d_b(D_1, D_2) = \inf_{\pi} \max_{p \in D_1} \norm{p - \pi(p)},
\end{equation}
where $\pi$ is a bijection between the points in $D_1$ and $D_2$, and
$\norm{\cdot}$ is the Euclidean distance. \todo{wrong definition, talk about allowed diagonal matching}
An alternative metric is the \emph{Wasserstein distance}, which generalizes the
bottleneck distance by computing the $p$-norm of the transport distance \todo{bad way to say this}:
\begin{equation}
    d_p(D_1, D_2) = \inf_{\pi} \left( \sum_{p \in D_1} \norm{p - \pi(p)}^p \right)^{1/p}.
\end{equation}

\section{Stability}

