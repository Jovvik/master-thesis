\chapter{Stable Generalized Density Functions}
\label{chap:stable_functions}

\todo{wording}
Chapter~\ref{chap:background} reviewed foundational stability results for GDFs.
This chapter examines the central question of this thesis: \emph{for which GDFs
$f(P, x)$ can we guarantee that the sublevel set filtration of $f_P$ is stable?}
We will explore several classes of GDFs that satisfy stability, ranging from
simple cases to more complex constructions, and also present examples of
functions that are not stable.

\section{0-stable generalized density functions}

The simplest class of stable GDFs are those that do not actually depend on the
input point cloud $P$.
\begin{definition}[Point Cloud Independent GDF]
    A GDF $f(P, x)$ is \emph{point cloud independent} if for any two point
    clouds $P, Q \subseteq \R^d$, we have $f(P, x) = f(Q, x)$ for all
    $x \in \R^d$.
\end{definition}
\todo{rewrite as theorem + proof}
The sublevel set filtrations of such functions are the same for all point
clouds, leading to identical persistence diagrams. It is known that the
bottleneck distance between identical persistence diagrams is
zero~\cite{edelsbrunner2010computational}, thus the condition for $0$-stability
is trivially satisfied:
\begin{equation}
    d_b(\dgm(f_P), \dgm(f_Q)) = 0 \leq 0 \cdot d_H(P, Q).
\end{equation}

This class does not include all $0$-stable GDFs, however. Consider the GDF
$f : \mathcal{P}(\R) \times \R \to \R$ defined by $f(P, x) = x + |P|$. The
sublevel sets $f_P^{-1}(-\infty, a]$ are exactly the intervals
$(-\infty, a - |P|]$, which always have exactly one connected component
that persists indefinitely. Thus, the persistence diagram of $f_P$ contains a
single point of dimension $0$ born at $-\infty$ that never dies, and is empty
in higher dimensions. Thus, for any two finite point clouds $P, Q \subseteq \R$,
the persistence diagrams $\dgm(f_P)$ and $\dgm(f_Q)$ are identical, and we have
$0$-stability without independence on the point cloud.
This example also trivially generalizes to $\R^d$ with roots of $f$ forming a
hyperplane.
\todo{terrible wording in this whole paragraph}
\todo{rewrite as lemma + proof}

\section{Point cloud Lipschitz generalized density functions}

A powerful sufficient condition for stability is a form of Lipschitz continuity:
\begin{definition}[Point Cloud Lipschitz (PC-Lipschitz) GDF]
    A GDF $f : \mathcal{P}(\R^d) \times \R^d \to \R$ is
    \emph{$c$-PC Lipschitz} if it is Lipschitz continuous with Lipschitz
    constant $c$ as a function of type $\mathcal{P}(\R^d) \to (\R^d \to \R)$,
    where the space of finite point clouds is equipped with the Hausdorff
    distance, and the space of real-valued functions $f_P : \R^d \to \R$ is
    equipped with the $L_\infty$-norm.
    \todo{is ``of type'' the right term here?}
\end{definition}
\begin{lemma}
    \label{lem:pc_lipschitz_stable}
    If $f$ is a $c$-PC Lipschitz GDF and its induced persistence modules are
    $q$-tame, then it is $c$-stable.
\end{lemma}
\begin{proof}
    Let $f$ be a $c$-PC Lipschitz GDF. By definition, for any two finite point
    clouds $P, Q \subseteq \R^d$, we have
    \begin{align}
        \frac{\norm{f_P - f_Q}_\infty}{d_H(P, Q)} & \leq c \\
        \norm{f_P - f_Q}_\infty & \leq c \cdot d_H(P, Q).
    \end{align}
    By Theorem~\ref{thm:stability_sup_norm}, the bottleneck distance between the
    persistence diagrams $\dgm(f_P)$ and $\dgm(f_Q)$ is bounded by the
    $L_\infty$-norm of the difference of the functions $f_P$ and $f_Q$.
    Combining these two inequalities, we obtain:
    \begin{equation}
        d_b(\dgm(f_P), \dgm(f_Q)) \leq \norm{f_P - f_Q}_\infty \leq c \cdot d_H(P, Q),
    \end{equation}
    which shows that $f$ is $c$-stable.
\end{proof}
While this lemma is a straightforward consequence of
Theorem~\ref{thm:stability_sup_norm}, it is nonetheless a powerful result.
To illustrate this, we apply it in the next sections.

\section{Weighted \v{C}ech filtration}

The stability of the weighted \v{C}ech filtration (and thus DTM-filtrations) was
stated in Theorem~\ref{thm:weighted_cech_stability}. We provide a more direct
proof of this result than in the original paper~\cite{anai2020dtm} by showing
that the corresponding GDF is PC-Lipschitz.

\begin{theorem}
    The GDF
    \begin{equation}
        f_{g, q}(P, x) = \min_{p \in P} (d(x, p)^{q} + g(p)^{q})^{1/q}
    \end{equation}
    for the weighted \v{C}ech filtration is $d$-stable with
    $d = (1 + c^q)^{1/q}$, where $c$ is the Lipschitz constant of the
    function $g : \R^d \to \R^+$.
\end{theorem}
\begin{proof}
    Let $P, Q \subseteq \R^d$ be two finite point clouds with Hausdorff distance
    $d_H(P, Q)$. Similarly to the proof of \v{C}ech stability using
    Theorem~\ref{thm:stability_sup_norm}, fix $x \in \R^d$ and let the minimum
    over $P$ be attained at $p \in P$. This implies that
    \begin{equation}
        \label{eq:stable_gdf_min}
        f_{g, q}(P, x) = (d(x, p)^q + g(p)^q)^{1/q}.
    \end{equation}
    By the definition of Hausdorff distance, there exists a point $q' \in Q$
    such that $d(p, q') \leq d_H(P, Q)$, which implies that
    \begin{equation}
        d(x, q') \leq d(x, p) + d(p, q') \leq d(x, p) + d_H(P, Q).
    \end{equation}
    By monotonicity of~\eqref{eq:stable_gdf_min} with respect to $d(x, p)$ and
    $g(p)$, we have
    \begin{align}
        f_{g, q}(Q, x)
        & = \min_{q'' \in Q} (d(x, q'')^q + g(q'')^q)^{1/q} \\
        & \leq (d(x, q')^q + g(q')^q)^{1/q} \\
        & \leq ((d(x, p) + d_H(P, Q))^q + g(q')^q)^{1/q} \\
        & \leq ((d(x, p) + d_H(P, Q))^q + (g(p) + c \cdot d_H(P, Q))^q)^{1/q} \\
        & = \norm{(d(x, p) + d_H(P, Q), g(p) + c \cdot d_H(P, Q))}_q.
    \end{align}
    Comparing the values of $f_{g, q}(P, x)$ and $f_{g, q}(Q, x)$, we have
    \begin{align}
        & f_{g, q}(Q, x) - f_{g, q}(P, x) \notag \\
        & \leq \norm{(d(x, p) + d_H(P, Q), g(p) + c \cdot d_H(P, Q))}_q - \norm{(d(x, p), g(p))}_q \\
        & \leq \norm{(d(x, p) + d_H(P, Q) - d(x, p), g(p) + c \cdot d_H(P, Q) - g(p))}_q \\
        & = \norm{(d_H(P, Q), c \cdot d_H(P, Q))}_q,
    \end{align}
    and by the same reasoning, we have
    \begin{equation}
        f_{g, q}(P, x) - f_{g, q}(Q, x) \leq \norm{(d_H(P, Q), c \cdot d_H(P, Q))}_q,
    \end{equation}
    which combined gives us
    \begin{align}
        \norm{f_{g, q}(P, \cdot) - f_{g, q}(Q, \cdot)}_\infty
        & = \sup_{x \in \R^d} |f_{g, q}(P, x) - f_{g, q}(Q, x)| \\
        & \leq \norm{(d_H(P, Q), c \cdot d_H(P, Q))}_q \\
        & = d_H(P, Q) \cdot \norm{(1, c)}_q \\
        & = d_H(P, Q) \cdot (1 + c^q)^{1/q},
    \end{align}
    which implies that
    \begin{equation}
        \norm{f_{g, q}(P, \cdot) - f_{g, q}(Q, \cdot)}_\infty \leq d_H(P, Q) \cdot (1 + c^q)^{1/q},
    \end{equation}
    which is exactly the condition for $d$-PC Lipschitz continuity with $d = (1 + c^q)^{1/q}$.

    As it has been already shown in (\cite{anai2020dtm}, Proposition 3.1),
    the induced persistence modules of the weighted \v{C}ech filtration are
    pointwise finite dimensional for finite point clouds $P$, which implies
    that they are $q$-tame. Thus, by Lemma~\ref{lem:pc_lipschitz_stable}, the
    weighted \v{C}ech filtration is $d$-stable with $d = (1 + c^q)^{1/q}$.
\end{proof}

\section{Generalized \v{C}ech filtrations}

The nearest neighbor distance function $f(P, x) = \min_{p \in P} d(x, p)$ can be
seen as taking a minimum over a set of functions $h_p(x) = d(x, p)$, each
centered around a point $p \in P$. We can generalize this to
$f(P, x) = \min_{p \in P} h(x, p)$, where $h(x, p) : \R^d \times \R^d \to \R$ is
a more general ``kernel'' function. We call such GDFs \emph{generalized \v{C}ech
filtrations}. The stability of such GDFs depends on the properties of the kernel
$h$.

\subsection{Isotropic, monotone and Lipschitz kernels}

Borrowing terminology from the field of kernel methods, we call a kernel
\emph{isotropic}
if it is only a function of the distance between $x$ and $p$
\cite{genton2001classes}:
\begin{definition}[Isotropic kernel]
    A kernel $h : \R^d \times \R^d \to \R$ is \emph{isotropic} if there exists a
    function $k : \R_+ \to \R$ such that $h(x, p) = k(d(x, p))$ for all
    $x, p \in \R^d$.
\end{definition}
Such kernels correspond to \v{C}ech filtrations where the radius of the growing
balls grows with the same rate for each point $p \in P$, but that rate changes
non-uniformly.

The generalized \v{C}ech filtration is stable if the kernel $h$ is isotropic,
monotone and Lipschitz continuous:
\begin{theorem}
    Let $h : \R^d \times \R^d \to \R$ be a kernel that is isotropic, monotone
    and Lipschitz continuous. Then the generalized \v{C}ech filtration using $h$
    is $c$-stable with $c$ being the Lipschitz constant of $h$.
\end{theorem}
\begin{proof}
    We want to show that the corresponding GDF is $c$-PC Lipschitz. Let $P, Q
    \subseteq \R^d$ be two finite point clouds. As in the previous proof, we
    fix a point $x \in \R^d$ and let the minimum over $P$ be attained at
    $p \in P$. Then we have $q' \in Q$ such that $d(p, q') \leq d_H(P, Q)$.
    \begin{align}
        f(Q, x) & = \min_{q \in Q} k(d(x, q)) \\
        & \leq k(d(x, q')) \\
        & \leq k(d(x, p) + d(p, q')) \\
        & \leq k(d(x, p) + d_H(P, Q)) \\
        & \leq k(d(x, p)) + c \cdot d_H(P, Q) \\
        & = f(P, x) + c \cdot d_H(P, Q),
    \end{align}
    and by the same reasoning, we have
    \begin{equation}
        f(P, x) \leq f(Q, x) + c \cdot d_H(P, Q).
    \end{equation}
    Combining these inequalities, we obtain
    \begin{align}
        \norm{f(P, \cdot) - f(Q, \cdot)}_\infty
        & = \sup_{x \in \R^d} |f(P, x) - f(Q, x)| \\
        & \leq c \cdot d_H(P, Q).
    \end{align}

    The induced persistence modules of $f(P, x)$ are $q$-tame by the same
    reasoning as in the proof of Lemma~\ref{lem:tameness_pfd}. Therefore, by
    Lemma~\ref{lem:pc_lipschitz_stable}, the generalized \v{C}ech
    filtration using $h$ is $c$-stable.
\end{proof}

\subsection{Kernels Lipschitz in $p$}

More generally, consider $f(P, x) = \min_{p \in P} h(x, p)$ where the kernel
$h(x, p)$ is $c$-Lipschitz with respect to its second argument $p$, for any
fixed $x$:
\begin{equation}
    |h(x, p_1) - h(x, p_2)| \leq c \cdot d(p_1, p_2).
\end{equation}
Such kernels correspond to generalized \v{C}ech filtrations where the standard
process of growing balls around points $p \in P$ is modified significantly;
the growing shapes may not be symmetric, connected, grow at the same rate for
different $p \in P$ or even include the point $p$ itself. The only restriction
imposed by Lipschitzness of the kernel is that as a point $p$ moves, the shape
changes in a controlled manner. Surprisingly, even this weak condition together
with a $q$-tameness condition is sufficient to guarantee stability of the
generalized \v{C}ech filtration.
\todo{rephrase this paragraph, explain more formally that ``shapes'' are just the sublevel sets}
\begin{theorem}
    Let $h : \R^d \times \R^d \to \R$ be a kernel that is Lipschitz continuous
    with respect to its second argument with Lipschitz constant $c$. Then the
    generalized \v{C}ech filtration using $h$ is $c$-stable if the induced
    persistence modules are $q$-tame.
\end{theorem}
\begin{proof}
    \todo{}
\end{proof}