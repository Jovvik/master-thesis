\begin{abstract}
    Topological data analysis (TDA) often uses density-like functions defined on
    the ambient space $\R^n$ to infer the underlying topological structure of a
    dataset $P \subseteq \R^n$. The persistent homology of the sublevelset or
    superlevelset filtrations induced by these functions captures multi-scale
    topological features. A key desirable property is \emph{stability}: small
    perturbations in the dataset result in similarly small changes in the
    persistence diagrams. A classic example is
    the nearest-neighbor distance function $f(x) \coloneqq \min_{p\in P} d(x,
    p)$. The bottleneck distance between persistence diagrams $\dgm_{f(P)}$ and
    $\dgm_{f(Q)}$ is bounded by the Hausdorff distance $d_H(P, Q)$ between
    datasets \cite{chazal2013persistencestabilitygeometriccomplexes}:
    \[d_b(\dgm_{f(P)}, \dgm_{f(Q)}) \leq d_H(P, Q).\]
    
    This thesis investigates \emph{generalized density functions} (GDFs), which are functions
    $f : \mathcal{P}(X) \times X \to \R$, and the conditions under which
    they satisfy a stability property of the form
    \[d_b(\dgm_{f(P)}, \dgm_{f(Q)}) \leq c \cdot d_H(P, Q)\]
    for some finite constant $c$.

    We identify several classes of GDFs for which such stability holds,
    including various generalizations of the nearest-neighbor distance function,
    Lipschitz functions and Morse functions. Furthermore, we examine the space
    of stable functions from an algebraic and topological perspective.
    Additionally, we identify conditions under which operations, such as
    addition, scaling and taking minima, preserve stability. Finally, we provide
    examples of generalized density functions for which stability is not
    preserved by these operations, demonstrating the limitations of combining
    stable functions.
    
    Our work unifies existing stability results with new insights into the
    properties and limitations of generalized density functions in TDA. These
    findings offer guidance for design and selection of GDFs, and point towards
    promising directions for future research.
\end{abstract}
