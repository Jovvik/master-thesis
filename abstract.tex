\begin{abstract}
    Topological data analysis (TDA) often uses density-like functions defined on
    the ambient space $\R^d$ to infer the underlying topological structure of a
    dataset $P \subseteq \R^d$. The persistent homology of the sublevelset or
    superlevelset filtrations induced by these functions captures multi-scale
    topological features. A key desirable property is \emph{stability}: small
    perturbations in the dataset result in similarly small changes in the
    persistence diagrams. A classic example is the nearest-neighbor distance
    function $f_{\operatorname{dist}}(P, x) \coloneqq \min_{p\in P} d(x, p)$,
    whose sublevel sets are homotopy equivalent to the \v{C}ech
    complex~\cite{schnider2024introduction}, for which the bottleneck
    distance between persistence diagrams $\dgm(f_P)$ and $\dgm(f_Q)$ is
    bounded by the Hausdorff distance $d_H(P, Q)$ between datasets
    \cite{chazal2013persistencestabilitygeometriccomplexes}:
    \[d_b(\dgm(f_P), \dgm(f_Q)) \leq d_H(P, Q).\]
    This thesis investigates \emph{generalized density functions}, which are
    functions $f : \mathcal{P}(\R^d) \times \R^d \to \R$, where
    $\mathcal{P}(\R^d)$ denotes the set of all finite subsets of $\R^d$, and
    the conditions under which they satisfy a stability property of the form
    \[d_b(\dgm(f_P), \dgm(f_Q)) \leq c \cdot d_H(P, Q)\]
    for some finite constant $c$.

    The primary contributions of this work are stability theorems for several
    classes of generalized density functions. Specifically, we prove stability
    bounds for:
    \begin{itemize}
        \item Several generalizations of the nearest-neighbor distance function
            of the form $f(P, x) = \min_{p\in P} h(x, p)$.
        \item Functions that are Lipschitz continuous with respect to the
            Hausdorff distance on the space of point clouds.
        \item Morse functions that satisfy a Lipschitz-like condition.
    \end{itemize}

    Beyond these core stability results, we explore the properties of the space
    of stable functions and investigate how common operations, such as addition
    and taking minima, affect stability. We identify conditions under which
    stability is preserved under these operations and provide counterexamples
    demonstrating cases where it is not.
    
    Our findings unify and extend existing stability results, offering practical
    guidance for the selection and design of generalized density functions for
    topological data analysis.
\end{abstract}
