% Some commands used in this file
\newcommand{\package}{\emph}

\chapter{Introduction}
\label{chap:introduction}

Topological Data Analysis (TDA) provides a robust framework for understanding
the shape of data~\cite{carlsson2021topological}. Among its tools is
\emph{persistent homology}, which extracts
topological features across different scales~\cite{boissonnat2018geometric}.
Traditionally, persistent homology is applied to a finite point cloud $P
\subseteq \R^d$ through filtrations of simplicial complexes built on top of $P$,
such as the Vietoris–Rips or \v{C}ech complexes~\cite{chazal2021introduction}.
An alternative (though sometimes equivalent) approach involves defining a
real-valued function $\R^d \to \R$ that encodes the geometry of the data, and
then computing the persistent homology of its sublevel (or superlevel) set
filtration~\cite{edelsbrunner2010computational}. The resulting persistent
homology captures topological features such as connected components, loops, and
voids, and encodes them in a persistence diagram, which summarizes their birth
and death across the filtration~\cite{chazal2021introduction}.

A classical and widely studied example is the \emph{nearest-neighbor distance
function},
\begin{equation}
    f(P, x) = \min_{p \in P} d(x, p),
\end{equation}
where $d$ is a metric on $\R^d$. The sublevel sets of this function are
homotopy equivalent to the \v{C}ech filtration~\cite{schnider2024introduction},
a fundamental construction in TDA.  This function satisfies a \emph{stability}
property~\cite{chazal2013persistencestabilitygeometriccomplexes}:
\begin{equation}
    d_b(\dgm(f_P), \dgm(f_Q)) \leq \cdot d_H(P, Q),
\end{equation}
where $d_b$ is the bottleneck distance between persistence diagrams and $d_H$ is
the Hausdorff distance between point clouds.
Stability ensures that small perturbations in the data lead to proportionally
small changes in the persistence diagrams, guaranteeing that the topological
features extracted from the data are robust to noise~\cite{chazal2021introduction}.

However, the nearest-neighbor function is just one of many possible density-like
functions for
TDA~\cite{anai2020dtm,hoefgeest2022christoffeldarbouxkerneltopologicaldata,phillips2015geometricinferencekerneldensity}.
In practice, alternative functions may offer computational
advantages~\cite{guibas2011witnessed,buchet2014efficientrobustpersistenthomology},
better capture intrinsic structure~\cite{anai2020dtm}, or incorporate domain
knowledge~\cite{fractalfract8120731}. For example, the DTM
filtration~\cite{anai2020dtm} modifies the nearest-neighbor function to make
it more robust to noise and outliers. This motivates the study of
\emph{generalized density functions} (GDFs), which are functions of the form
\begin{equation}
    f(P, x) : \mathcal{P}(\R^d) \times \R^d \to \R.
\end{equation}
For a given point cloud $P$, such a function defines a real-valued function
$f_P: \R^d \to \R$, whose sublevel set filtration can be used to analyze the
topological features of $P$. We say $f$ is $c$-\emph{stable} if for all finite
point clouds $P, Q \subseteq \R$ we have
\begin{equation}
    d_b(\dgm(f_P), \dgm(f_Q)) \leq c \cdot d_H(P, Q),
\end{equation}
where the \emph{stability constant} $c$ measures the sensitivity of the
persistence diagrams to perturbations in the data.

We limit ourselves to the case where $P$ and $Q$ are finite sets, as known
stability results for GDFs often hold only in this case, because otherwise the
functions $f_P$ are not tame, and, consequently, the persistence diagrams
$\dgm(f_P)$ are not $q$-tame.

We also only consider the space $\R^d$ instead of a general topological space
for the sake of simplicity, as the various theorems used in this thesis impose
different conditions on the underlying topological space. An inquisitive reader
is invited to track down the most general conditions for the various novel
theorems by keeping track of the theorems used in the proofs.

In this thesis, we primarily investigate which GDFs are
stable. We prove stability of the following classes of functions:
\begin{enumerate}
    \item Functions of the form $f(P, x) = \min_{p \in P} h(d(p, x))$, where $h$
          is a monotone Lipschitz continuous function. These generalize the
          classical nearest-neighbor function while preserving stability.
    \item Further generalization to $f(P, x) = \min_{p \in P} h(p, x)$, where
          $h$ is a Lipschitz function with respect to $p$.
    \item Functions $f(P, x)$ that are Lipschitz with respect to point clouds
          \emph{(PC-Lipschitz)}, where $\mathcal{P}(\R^d)$ is equipped with the
          Hausdorff distance, and the space of real-valued functions $f_P : \R^d
          \to \R$ is equipped with the $L_\infty$-norm.
    \item Finally, we consider Morse functions whose restriction to the critical
          points is Lipschitz continuous.
\end{enumerate}

Beyond these classes, we also study when stability is preserved under common
operations. For example, if two functions are each stable, under what conditions
is their pointwise minimum or average also stable? We identify sufficient
conditions and provide concrete counterexamples demonstrating cases where
stability fails. This analysis shows how more complex stable functions can be
constructed from simpler ones, and provides insight into the structure of the
space of stable functions.

Finally, we examine the topological and algebraic properties of the space of
$c$-stable functions itself. We analyze properties such as convexity,
contractibility, and closedness, as well as establish a lattice structure on the
set of $c$-PC-Lipschitz functions.

Taken together, our results contribute to the theoretical foundations of TDA by
deepening the understanding of stability for generalized density functions. They
also provide practical guidance for designing stable filtrations.

\todo{this is too similar to the abstract --- add more context?}

\todo{Say something about the choice of $\R^d$ and finite subsets}