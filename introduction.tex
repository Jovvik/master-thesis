% Some commands used in this file
\newcommand{\package}{\emph}

\chapter{Introduction}
\label{chap:introduction}

Topological Data Analysis (TDA) provides a robust framework for understanding
the shape of data~\cite{carlsson2021topological}. Among its tools is
\emph{persistent homology}, which extracts
topological features across different scales~\cite{boissonnat2018geometric}.
Traditionally, persistent homology is applied to a finite point cloud $P
\subseteq \R^d$ through filtrations of simplicial complexes built on top of $P$,
such as the Vietoris–Rips or \v{C}ech complexes~\cite{chazal2021introduction}.
An alternative (though sometimes equivalent) approach involves defining a
real-valued function $\R^d \to \R$ that encodes the geometry of the data, and
then computing the persistent homology of its sublevel (or superlevel) set
filtration~\cite{edelsbrunner2010computational}. The resulting persistent
homology captures topological features such as connected components, loops, and
voids, and encodes them in a persistence diagram, which summarizes their birth
and death across the filtration~\cite{chazal2021introduction}.

A classical and widely studied example is the \emph{nearest-neighbor distance
function},
\begin{equation}
    f(P, x) = \min_{p \in P} d(x, p),
\end{equation}
where $d$ is a metric on $\R^d$. The sublevel sets of this function are
homotopy equivalent to the \v{C}ech filtration~\cite{schnider2024introduction},
a fundamental construction in TDA.  This function satisfies a \emph{stability}
property~\cite{chazal2013persistencestabilitygeometriccomplexes}:
\begin{equation}
    d_b(\dgm(f_P), \dgm(f_Q)) \leq c \cdot d_H(P, Q),
\end{equation}
where $d_b$ is the bottleneck distance between persistence diagrams and $d_H$ is
the Hausdorff distance between point clouds. Stability is not merely a
theoretical nicety; it is the foundation that provides reliability of
TDA~\cite{chazal2021introduction}. In real-world applications, where data is
inevitably noisy or subject to measurement error, an unstable method could
produce drastically different topological summaries from minor, inconsequential
variations in the input. A stable GDF ensures that the extracted topological
features are genuine reflections of the data's underlying structure, rather than
artifacts of noise or sampling.

While the nearest-neighbor function is stable, it is just one of many possible
density-like functions for
TDA~\cite{anai2020dtm,hoefgeest2022christoffeldarbouxkerneltopologicaldata,phillips2015geometricinferencekerneldensity}.
In practice, alternative functions may offer computational
advantages~\cite{guibas2011witnessed,buchet2014efficientrobustpersistenthomology},
better capture intrinsic structure~\cite{anai2020dtm}, or incorporate domain
knowledge~\cite{fractalfract8120731}. For example, the DTM
filtration~\cite{anai2020dtm} modifies the nearest-neighbor function to make
it more robust to noise and outliers. This motivates the study of
\emph{generalized density functions} (GDFs), which are functions of the form
\begin{equation}
    f(P, x) : \mathcal{P}(\R^d) \times \R^d \to \R.
\end{equation}
For a given point cloud $P$, such a function defines a real-valued function
$f_P: \R^d \to \R$, whose sublevel set filtration can be used to analyze the
topological features of $P$. We say $f$ is $c$-\emph{stable} if for all finite
point clouds $P, Q \subseteq \R^d$ we have
\begin{equation}
    d_b(\dgm(f_P), \dgm(f_Q)) \leq c \cdot d_H(P, Q),
\end{equation}
where the \emph{stability constant} $c$ measures the sensitivity of the
persistence diagrams to perturbations in the data.

We limit ourselves to the case where $P$ and $Q$ are finite sets, as known
stability results for GDFs often hold only in this case, because otherwise the
functions $f_P$ often are not tame, and, consequently, the persistence diagrams
$\dgm(f_P)$ are not $q$-tame.

We also only consider the space $\R^d$ instead of a general topological space
for the sake of simplicity, as the various theorems used in this thesis impose
different conditions on the underlying topological space. The euclidean space
$\R^d$ was chosen as the lowest common denominator, although many results in
this thesis hold for more general spaces. An inquisitive reader is invited to
track down the most general conditions for the various novel theorems by keeping
track of the theorems used in the proofs.

In this thesis, we primarily investigate which GDFs are stable. Existing
stability results, such as those for the \v{C}ech filtration and its weighted
variants, provide a starting point. This thesis aims to create a more
comprehensive framework by identifying broader conditions that ensure stability,
thereby unifying and extending these results. Concretely, we prove stability of
the following classes of functions:
\begin{enumerate}
    \item We begin by examining natural extensions of the nearest-neighbor
        function, specifically of the form $f(P, x) = \min_{p \in P} h(d(p,
        x))$, where $h$ is a monotone Lipschitz continuous function. This
        is a modification of the \v{C}ech complex which allows the balls to grow
        at non-uniform rates, although it still requires the growth rate to be
        the same for all points $p \in P$.
    \item This is further extended to $f(P, x) = \min_{p \in P} h(p, x)$, where
        $h(p, x)$ is a Lipschitz function with respect to $p$. This allows for
        more general shapes, as well as different growth rates for different
        points $p$.
    \item We then investigate a more direct condition: functions $f(P, x)$ that
        are Lipschitz with respect to point clouds \emph{(PC-Lipschitz)}, where
        $\mathcal{P}(\R^d)$ is equipped with the Hausdorff distance, and the
        space of real-valued functions $f_P : \R^d \to \R$ is equipped with the
        $L_\infty$-norm. This class includes the previous two classes as
        special cases, and allows for more general functions that are still
        stable, but may not be represented as a minimum over a set of
        functions.
    \item Finally, recognizing that full Lipschitz continuity might be too
        strong, we consider Morse functions where Lipschitz-like conditions are
        imposed only on critical values. This aims to capture stability for
        functions whose may not be Lipschitz continuous globally, but are
        nonetheless stable.
\end{enumerate}

Beyond these classes, we also study when stability is preserved under common
operations. For example, if two functions are each stable, under what conditions
is their pointwise minimum or average also stable? We identify sufficient
conditions and provide concrete counterexamples demonstrating cases where
stability fails. This analysis shows how more complex stable functions can be
constructed from simpler ones, and provides insight into the structure of the
space of stable functions.

Finally, we examine the topological and algebraic properties of the space of
$c$-stable functions itself. We analyze properties such as convexity,
contractibility, and closedness, as well as establish a lattice structure on the
set of $c$-PC-Lipschitz functions.

Taken together, our results contribute to the theoretical foundations of TDA by
deepening the understanding of stability for generalized density functions. They
also provide practical guidance for designing stable filtrations.