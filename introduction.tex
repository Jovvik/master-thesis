% Some commands used in this file
\newcommand{\package}{\emph}

\chapter{Introduction}
\label{chap:introduction}

Topological Data Analysis (TDA) provides a robust framework for understanding
the shape of data. Among its tools is \emph{persistent homology}, which extracts
topological features across different scales. Traditionally, persistent homology
is applied to a finite point cloud $P \subseteq X$ through filtrations of
simplicial complexes built on top of $P$, such as the Vietoris–Rips or \v{C}ech
complexes. An alternative (though sometimes equivalent) approach involves
defining a real-valued function $X \to \R$ that encodes the geometry of the
data, and then computing the persistent homology of its sublevel (or superlevel)
set filtration. The resulting persistent homology captures topological features
such as connected components, loops, and voids, and encodes them in a
persistence diagram, which summarizes their birth and death across the
filtration.

A classical and widely studied example is the \emph{nearest-neighbor distance
function},
\begin{equation}
    f(P, x) = \min_{p \in P} d(x, p),
\end{equation}
where $d$ is a metric on $X$. The homology groups of the sublevel set filtration
of this function are precisely the homology groups of the \v{C}ech filtration.
This function satisfies a \emph{stability} property:
\begin{equation}
    d_b(\dgm_{f(P)}, \dgm_{f(Q)}) \leq d_H(P, Q),
\end{equation}
where $d_b$ is the bottleneck distance between persistence diagrams and $d_H$ is
the Hausdorff distance between point clouds.
Stability ensures that small perturbations in the data lead to proportionally
small changes in the persistence diagrams, guaranteeing that the topological
features extracted from the data are robust to noise.

However, the nearest-neighbor function is just one of many possible density-like
functions for TDA. In practice, alternative functions may offer computational
advantages, better capture intrinsic structure, or incorporate domain knowledge.
For example, the DTM filtration~\cite{anai2020dtm} modifies the nearest-neighbor
function to make it more robust to noise and outliers. This motivates the study
of \emph{generalized density functions}, which are functions of the form
\begin{equation}
    f(P, x) : \mathcal{P}(X) \times X \to \R.
\end{equation}
For a given point cloud $P$, such a function defines a real-valued function
$f_P: X \to \R$, whose sublevel set filtration can be used to analyze the
topological features of $P$. We say $f$ is $c$-\emph{stable} if for all point
clouds $P, Q \subseteq X$ we have
\begin{equation}
    d_b(\dgm_{f(P)}, \dgm_{f(Q)}) \leq c \cdot d_H(P, Q),
\end{equation}
where the constant $c$ measures the sensitivity of the persistence diagrams to
perturbations in the data.

In this thesis, we investigate which functions are stable. We consider the
following classes of functions:
\begin{enumerate}
    \item Functions of the form $f(P, x) = \min_{p \in P} h(d(p, x))$, where $h$
          is a monotone Lipschitz continuous function. These generalize the
          classical nearest-neighbor function while preserving stability.
    \item Further generalization to $f(P, x) = \min_{p \in P} h(p, x)$, where
          $h$ is a Lipschitz function with respect to $p$.
    \item Functions $f(P, x)$ that are Lipschitz with respect to point clouds
          \emph{(PC-Lipschitz)}, where $\mathcal{P}(X)$ is equipped with the
          Hausdorff distance, and the space of real-valued functions $f_P : X
          \to \R$ is equipped with the $L_\infty$-norm.
    \item Finally, we consider Morse functions whose restriction to the critical
          points is Lipschitz continuous.
\end{enumerate}

Beyond these classes, we also study when stability is preserved under common
operations. For example, if two functions are each stable, under what conditions
is their pointwise minimum or average also stable? We identify sufficient
conditions and provide concrete counterexamples demonstrating cases where
stability fails. This analysis shows how more complex stable functions can be
constructed from simpler ones, and provides insight into the structure of the
space of stable functions.

Finally, we examine the topological and algebraic properties of the space of
$c$-stable functions itself. We analyze properties such as convexity,
contractibility, and closedness, as well as establish a lattice structure on the
set of $c$-PC-Lipschitz functions.

Taken together, our results contribute to the theoretical foundations of TDA by
deepening the understanding of stability for generalized density functions. They
also provide practical guidance for designing stable filtrations.

\section{Overview}

The remainder of this thesis is organized as follows:
\begin{itemize}
    \item Chapter~\ref{chap:background} provides the necessary background on
        persistent homology, stability, as well as known stability results.
    \item Chapter~\ref{chap:stable_functions} presents our main contributions:
        stability theorems for various classes of generalized density functions.
    \item Chapter~\ref{chap:operations} studies stability under common
        operations.
    \item Chapter~\ref{chap:space} explores the properties of the space of
        $c$-stable functions, including its topological and algebraic
        structure.
    \item Chapter~\ref{chap:conclusion} concludes with directions for future
        work and open questions.
\end{itemize}
